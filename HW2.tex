\documentclass{article}\usepackage[]{graphicx}\usepackage[]{xcolor}
% maxwidth is the original width if it is less than linewidth
% otherwise use linewidth (to make sure the graphics do not exceed the margin)
\makeatletter
\def\maxwidth{ %
  \ifdim\Gin@nat@width>\linewidth
    \linewidth
  \else
    \Gin@nat@width
  \fi
}
\makeatother

\definecolor{fgcolor}{rgb}{0.345, 0.345, 0.345}
\newcommand{\hlnum}[1]{\textcolor[rgb]{0.686,0.059,0.569}{#1}}%
\newcommand{\hlsng}[1]{\textcolor[rgb]{0.192,0.494,0.8}{#1}}%
\newcommand{\hlcom}[1]{\textcolor[rgb]{0.678,0.584,0.686}{\textit{#1}}}%
\newcommand{\hlopt}[1]{\textcolor[rgb]{0,0,0}{#1}}%
\newcommand{\hldef}[1]{\textcolor[rgb]{0.345,0.345,0.345}{#1}}%
\newcommand{\hlkwa}[1]{\textcolor[rgb]{0.161,0.373,0.58}{\textbf{#1}}}%
\newcommand{\hlkwb}[1]{\textcolor[rgb]{0.69,0.353,0.396}{#1}}%
\newcommand{\hlkwc}[1]{\textcolor[rgb]{0.333,0.667,0.333}{#1}}%
\newcommand{\hlkwd}[1]{\textcolor[rgb]{0.737,0.353,0.396}{\textbf{#1}}}%
\let\hlipl\hlkwb

\usepackage{framed}
\makeatletter
\newenvironment{kframe}{%
 \def\at@end@of@kframe{}%
 \ifinner\ifhmode%
  \def\at@end@of@kframe{\end{minipage}}%
  \begin{minipage}{\columnwidth}%
 \fi\fi%
 \def\FrameCommand##1{\hskip\@totalleftmargin \hskip-\fboxsep
 \colorbox{shadecolor}{##1}\hskip-\fboxsep
     % There is no \\@totalrightmargin, so:
     \hskip-\linewidth \hskip-\@totalleftmargin \hskip\columnwidth}%
 \MakeFramed {\advance\hsize-\width
   \@totalleftmargin\z@ \linewidth\hsize
   \@setminipage}}%
 {\par\unskip\endMakeFramed%
 \at@end@of@kframe}
\makeatother

\definecolor{shadecolor}{rgb}{.97, .97, .97}
\definecolor{messagecolor}{rgb}{0, 0, 0}
\definecolor{warningcolor}{rgb}{1, 0, 1}
\definecolor{errorcolor}{rgb}{1, 0, 0}
\newenvironment{knitrout}{}{} % an empty environment to be redefined in TeX

\usepackage{alltt}
\usepackage[margin=1.0in]{geometry} % To set margins
\usepackage{amsmath}  % This allows me to use the align functionality.
                      % If you find yourself trying to replicate
                      % something you found online, ensure you're
                      % loading the necessary packages!
\usepackage{amsfonts} % Math font
\usepackage{fancyvrb}
\usepackage{hyperref} % For including hyperlinks
\usepackage[shortlabels]{enumitem}% For enumerated lists with labels specified
                                  % We had to run tlmgr_install("enumitem") in R
\usepackage{float}    % For telling R where to put a table/figure
\usepackage{natbib}        %For the bibliography
\bibliographystyle{apalike}%For the bibliography
\IfFileExists{upquote.sty}{\usepackage{upquote}}{}
\begin{document}

\begin{enumerate}
%%%%%%%%%%%%%%%%%%%%%%%%%%%%%%%%%%%%%%%%%%%%%%%%%%%%%%%%%%%%%%%%%%%%%%%%%%%%%%%%
%%%%%%%%%%%%%%%%%%%%%%%%%%%%%%%%%%%%%%%%%%%%%%%%%%%%%%%%%%%%%%%%%%%%%%%%%%%%%%%%
% QUESTION 1
%%%%%%%%%%%%%%%%%%%%%%%%%%%%%%%%%%%%%%%%%%%%%%%%%%%%%%%%%%%%%%%%%%%%%%%%%%%%%%%%
%%%%%%%%%%%%%%%%%%%%%%%%%%%%%%%%%%%%%%%%%%%%%%%%%%%%%%%%%%%%%%%%%%%%%%%%%%%%%%%%
\item Consider the following integral:
\[\int_{a}^{b} (7-x^2)dx.\]
While this is a relatively straightforward integral -- split the difference and use power rule for integration -- we can approximate the area under this curve using Riemann sums. Below, I describe four rules for computing Riemann sums using rectangles of width $\delta_x$.
\begin{itemize}
  \item Left Rule uses the left points to create the rectangles.
  \[\text{Area} = \delta_x (f(a) + f(a + \delta_x) + f(a +2\delta_x) + \cdots + f(b-\delta_x))\]
  \item Right Rule uses the right points to create the rectangles.
  \[\text{Area} = \delta_x (f(a + \delta_x) + f(a +2\delta_x) + \cdots + f(b-\delta_x) + f(b))\]
  \item Midpoint Rule uses the midpoints between the left and right points to create the rectangles.
  \[\text{Area} = \delta_x \left(f\left(a + \frac{\delta_x}{2}\right) + f\left(a + \frac{3\delta_x}{2}\right) + \cdots +  f\left(b - \frac{\delta_x}{2}\right)\right)\]
  \item Trapezoidial Rule averages the rectangles created using left and right endpoints, which results in areas of trapezoids.
  \[\text{Area} = \frac{1}{2} \delta_x \left(f(a) + 2f(a + \delta_x) + 2f(a+2\delta_x) + \cdots + f(b)\right)\]
\end{itemize}
The first step, is to create a function that computes the integrand.
\begin{knitrout}\scriptsize
\definecolor{shadecolor}{rgb}{0.969, 0.969, 0.969}\color{fgcolor}\begin{kframe}
\begin{alltt}
\hldef{integrand} \hlkwb{<-} \hlkwa{function}\hldef{(}\hlkwc{x}\hldef{)\{}
  \hldef{f} \hlkwb{<-} \hlnum{7} \hlopt{-} \hlnum{2} \hlopt{*} \hldef{x}\hlopt{^}\hlnum{2}
  \hlkwd{return}\hldef{(f)}
\hldef{\}}
\end{alltt}
\end{kframe}
\end{knitrout}
Next, I set up an example choice of $a$, $b$, and the number of rectangles.
\begin{knitrout}\scriptsize
\definecolor{shadecolor}{rgb}{0.969, 0.969, 0.969}\color{fgcolor}\begin{kframe}
\begin{alltt}
\hldef{a} \hlkwb{<-} \hlnum{0}
\hldef{b} \hlkwb{<-} \hlnum{2}
\hldef{n.rect} \hlkwb{<-} \hlnum{100}
\hldef{(delta.x} \hlkwb{<-} \hldef{(b}\hlopt{-}\hldef{a)}\hlopt{/}\hldef{n.rect)}
\end{alltt}
\begin{verbatim}
## [1] 0.02
\end{verbatim}
\end{kframe}
\end{knitrout}
We can compute the area using the Left Rule as follows.
\begin{knitrout}\scriptsize
\definecolor{shadecolor}{rgb}{0.969, 0.969, 0.969}\color{fgcolor}\begin{kframe}
\begin{alltt}
\hldef{left.points} \hlkwb{<-} \hldef{a} \hlopt{+} \hlnum{0}\hlopt{:}\hlnum{99}\hlopt{*}\hldef{(delta.x)}
\hldef{(left.area} \hlkwb{<-} \hlkwd{sum}\hldef{(delta.x}\hlopt{*}\hldef{(}\hlkwd{integrand}\hldef{(left.points))))}
\end{alltt}
\begin{verbatim}
## [1] 8.7464
\end{verbatim}
\end{kframe}
\end{knitrout}
We can compute the area using the Right Rule as follows.
\begin{knitrout}\scriptsize
\definecolor{shadecolor}{rgb}{0.969, 0.969, 0.969}\color{fgcolor}\begin{kframe}
\begin{alltt}
\hldef{right.points} \hlkwb{<-} \hldef{a} \hlopt{+} \hlnum{1}\hlopt{:}\hlnum{100}\hlopt{*}\hldef{(delta.x)}
\hldef{(right.area} \hlkwb{<-} \hlkwd{sum}\hldef{(delta.x}\hlopt{*}\hldef{(}\hlkwd{integrand}\hldef{(right.points))))}
\end{alltt}
\begin{verbatim}
## [1] 8.5864
\end{verbatim}
\end{kframe}
\end{knitrout}
We can compute the area using the Midpoint Rule as follows.
\begin{knitrout}\scriptsize
\definecolor{shadecolor}{rgb}{0.969, 0.969, 0.969}\color{fgcolor}\begin{kframe}
\begin{alltt}
\hldef{mid.points} \hlkwb{<-} \hldef{(left.points}\hlopt{+}\hldef{right.points)}\hlopt{/}\hlnum{2}
\hldef{(mid.area} \hlkwb{<-} \hlkwd{sum}\hldef{(delta.x}\hlopt{*}\hldef{(}\hlkwd{integrand}\hldef{(mid.points))))}
\end{alltt}
\begin{verbatim}
## [1] 8.6668
\end{verbatim}
\end{kframe}
\end{knitrout}
\newpage
\begin{enumerate}
  \item Write code that computes the area using the Trapezoidial Rule.\\
  \textbf{Solution:}
\begin{knitrout}\scriptsize
\definecolor{shadecolor}{rgb}{0.969, 0.969, 0.969}\color{fgcolor}\begin{kframe}
\begin{alltt}
\hldef{trap.points} \hlkwb{<-} \hldef{a} \hlopt{+} \hlnum{1}\hlopt{:}\hlnum{99}\hlopt{*}\hldef{(delta.x)}
\hldef{(trap.area} \hlkwb{<-} \hlnum{1}\hlopt{/}\hlnum{2}\hlopt{*}\hldef{delta.x}\hlopt{*}\hldef{(}\hlkwd{integrand}\hldef{(a))} \hlopt{+} \hlkwd{sum}\hldef{(}\hlnum{1}\hlopt{/}\hlnum{2}\hlopt{*}\hldef{delta.x}\hlopt{*}\hldef{(}\hlnum{2}\hlopt{*}\hlkwd{integrand}\hldef{(trap.points)))} \hlopt{+} \hlnum{1}\hlopt{/}\hlnum{2}\hlopt{*}\hldef{delta.x}\hlopt{*}\hldef{(}\hlkwd{integrand}\hldef{(b)))}
\end{alltt}
\begin{verbatim}
## [1] 8.6664
\end{verbatim}
\end{kframe}
\end{knitrout}
  \item Write a function that takes \texttt{a}, \texttt{b}, and number of 
  rectangles (\texttt{n.rect}) in as input and returns the Trapezoidial Rule 
  by default, but can return left, right, or midpoint when necessary. Use the
  skeleton below. When you are done, remove \texttt{eval=FALSE} to show that
  the function provides the expected result for the example above.\\
  \textbf{Solution:}
\begin{knitrout}\scriptsize
\definecolor{shadecolor}{rgb}{0.969, 0.969, 0.969}\color{fgcolor}\begin{kframe}
\begin{alltt}
\hldef{riemann.sums} \hlkwb{<-} \hlkwa{function}\hldef{(}\hlkwc{fnct}\hldef{,}                        \hlcom{# function to integrate}
                         \hlkwc{a}\hldef{,}                           \hlcom{# lower bound of integral}
                         \hlkwc{b}\hldef{,}                           \hlcom{# upper bound of integral}
                         \hlkwc{n.rect}\hldef{,}                      \hlcom{# number of  bound of integral}
                         \hlkwc{method} \hldef{=} \hlsng{"Trapezoidial"}\hldef{)\{}    \hlcom{# method to use (trap by default)}
  \hlcom{######################################}
  \hlcom{# Check Input}
  \hlcom{######################################}
  \hlkwa{if}\hldef{(}\hlopt{!}\hlkwd{is.numeric}\hldef{(a))\{} \hlcom{# if a is not numeric}
    \hlkwd{stop}\hldef{(}\hlsng{"The lower bound of the integral (a) must be numeric."}\hldef{)}
  \hldef{\}}
  \hlkwa{if}\hldef{(}\hlopt{!}\hlkwd{is.numeric}\hldef{(b))\{} \hlcom{# if b is not numeric}
    \hlkwd{stop}\hldef{(}\hlsng{"The lower bound of the integral (a) must be numeric."}\hldef{)}
  \hldef{\}}
  \hlkwa{if}\hldef{(}\hlopt{!}\hldef{(}\hlkwd{is.numeric}\hldef{(n.rect))} \hlopt{|} \hldef{(n.rect}\hlopt\hlnum{1}\hlopt{!=}\hlnum{0}\hldef{))\{} \hlcom{# if n.rect is not a whole number}
    \hlkwd{stop}\hldef{(}\hlsng{"The number of rectangles must be a positive whole number."}\hldef{)}
  \hldef{\}}
  \hlcom{######################################}
  \hlcom{# Compute Area}
  \hlcom{######################################}
  \hldef{area} \hlkwb{<-} \hlnum{0}
  \hldef{delta.x} \hlkwb{<-} \hldef{(b}\hlopt{-}\hldef{a)}\hlopt{/}\hldef{n.rect}
  \hldef{end.point} \hlkwb{<-} \hldef{(n.rect}\hlopt{-}\hlnum{1}\hldef{)}
  \hlkwa{if}\hldef{(method} \hlopt{==} \hlsng{"Left"}\hldef{)\{}
    \hldef{left.points} \hlkwb{<-} \hldef{a} \hlopt{+} \hlnum{0}\hlopt{:}\hldef{end.point}\hlopt{*}\hldef{(delta.x)}
    \hldef{area} \hlkwb{<-} \hlkwd{sum}\hldef{(delta.x}\hlopt{*}\hldef{(}\hlkwd{fnct}\hldef{(left.points)))}
  \hldef{\}}\hlkwa{else if}\hldef{(method} \hlopt{==} \hlsng{"Right"}\hldef{)\{}
    \hldef{right.points} \hlkwb{<-} \hldef{a} \hlopt{+} \hlnum{1}\hlopt{:}\hldef{n.rect}\hlopt{*}\hldef{(delta.x)}
    \hldef{area} \hlkwb{<-} \hlkwd{sum}\hldef{(delta.x}\hlopt{*}\hldef{(}\hlkwd{fnct}\hldef{(right.points)))}
  \hldef{\}}\hlkwa{else if}\hldef{(method} \hlopt{==} \hlsng{"Midpoint"}\hldef{)\{}
    \hldef{mid.points} \hlkwb{<-} \hldef{(left.points}\hlopt{+}\hldef{right.points)}\hlopt{/}\hlnum{2}
    \hldef{area} \hlkwb{<-} \hlkwd{sum}\hldef{(delta.x}\hlopt{*}\hldef{(}\hlkwd{fnct}\hldef{(mid.points)))}
  \hldef{\}}\hlkwa{else if}\hldef{(method} \hlopt{==} \hlsng{"Trapezoidial"}\hldef{)\{}
    \hldef{trap.points} \hlkwb{<-} \hldef{a} \hlopt{+} \hlnum{1}\hlopt{:}\hldef{end.point}\hlopt{*}\hldef{(delta.x)}
    \hldef{area} \hlkwb{<-} \hlnum{1}\hlopt{/}\hlnum{2}\hlopt{*}\hldef{delta.x}\hlopt{*}\hldef{(}\hlkwd{fnct}\hldef{(a))} \hlopt{+} \hlkwd{sum}\hldef{(}\hlnum{1}\hlopt{/}\hlnum{2}\hlopt{*}\hldef{delta.x}\hlopt{*}\hldef{(}\hlnum{2}\hlopt{*}\hlkwd{fnct}\hldef{(trap.points)))} \hlopt{+} \hlnum{1}\hlopt{/}\hlnum{2}\hlopt{*}\hldef{delta.x}\hlopt{*}\hldef{(}\hlkwd{fnct}\hldef{(b))}
  \hldef{\}}\hlkwa{else}\hldef{\{}
    \hlkwd{stop}\hldef{(}\hlsng{"Please select a valid method (e.g., 'Left', 'Right', 'Midpoint', 'Trapezoidial')"}\hldef{)}
  \hldef{\}}
  \hlcom{######################################}
  \hlcom{# Return the area}
  \hlcom{######################################}
  \hlkwd{return}\hldef{(area)}
\hldef{\}}
\hlcom{######################################}
\hlcom{# Test the function}
\hlcom{######################################}
\hlkwd{riemann.sums}\hldef{(}\hlkwc{fnct} \hldef{= integrand,}
             \hlkwc{a} \hldef{=} \hlnum{0}\hldef{,}
             \hlkwc{b} \hldef{=} \hlnum{2}\hldef{,}
             \hlkwc{n.rect} \hldef{=} \hlnum{100}\hldef{)}
\end{alltt}
\begin{verbatim}
## [1] 8.6664
\end{verbatim}
\begin{alltt}
\hlcom{######################################}
\hlcom{# Compare to numerical integral}
\hlcom{######################################}
\hlkwd{integrate}\hldef{(}\hlkwc{f} \hldef{= integrand,} \hlcom{# integrate() is an R function}
          \hlkwc{lower} \hldef{=} \hlnum{0}\hldef{,}     \hlcom{# that completes numerical}
          \hlkwc{upper} \hldef{=} \hlnum{2}\hldef{)}     \hlcom{# integration}
\end{alltt}
\begin{verbatim}
## 8.666667 with absolute error < 9.8e-14
\end{verbatim}
\end{kframe}
\end{knitrout}
\end{enumerate}
\end{enumerate}

\bibliography{bibliography}
\end{document}
